\documentclass[a4paper, 10pt]{article}
\usepackage{physics}
\usepackage[english, russian]{babel}
\usepackage[utf8x]{inputenc} 
\usepackage{mathrsfs}
\usepackage{geometry}
\usepackage{graphicx}
\geometry{
  a4paper,
  top=25mm, 
  right=15mm, 
  bottom=25mm, 
  left=30mm
}
\title{Уравнение Линдблада для двухуровневой системы, взаимодействующей с термостатом}
\author{Pan Vyacheslav Igorevich}
\date{\today}
\begin{document}
\maketitle
    \begin{abstract}
        Уравнение Шредингера (\ref{link name}), широко применяемое для нахождения волновой функции, имеет ограниченное применение, 
        так как, описывая изменение системы только под действием потенциальных сил, позволяет определить только чистые 
        состояния\footnote{Полностью известное квантовое состояние.} и не способно описать диссипатицию\footnote{Необратимая потеря энергии.}
        квантовой системы.

        \begin{equation} \label{link name}
            i\hbar\partial_{t} \ket{\psi} = \hat{H}\ket{\psi}
        \end{equation}

        В то же время матрица плотности может задавать как чистые, так и смешанные состояния. Уравнение Линдблада
        (\ref{link name}), рассматривоемое в данной работе, является уравнением матрицы плотности, описывающим ее эволюцию.

        \begin{equation}\label{link name}
            \partial_t \rho = -\frac{i}{\hbar} [H,\rho] + \sum_i \gamma_i ( L_{i} \rho L^{\dagger} - \frac{1}{2} [L_{i}^{\dagger} L_i , \rho]) 
        \end{equation}

    \end{abstract}

    \section{Введение}
        Введем некоторые постулаты квантовой механики для чистых состояний.
        \\ \\
        \begin{itshape}
            \centering Постулат 1. С любой закрытой квантовой системой связано конечномерное или бесконечномерное 
            Гильбертово пространство\footnote{Линейное пространство, в котором норма порождается скалярным произведеднием.} $\mathscr{H}$ 
            над полем комплексных чисел, 
            которому принадлежит вектор состояний ($\ket{\psi} \in \mathscr{H}$).
        \end{itshape}
        \\ \\
        Состояния системы, описываемые векторами состояний называют чистыми. Зная вектор состояния системы, мы владеем наибольшей возможной 
        информацией о ней.
        Вектор состояния $\Psi$ в нотации Дирака можно записать как 
     
        \begin{equation}\label{link name}
            \Psi = \sum_i a_i \ket{\psi_i}
        \end{equation}
        где $\psi_i$ --- возможное состояние системы, $a_i$ --- амплитуда вероятности нахождения системы в состоянии с индексом i. Так как суммарная 
        вероятность всех состояний должна ровняться единице,
        
        \begin{equation}\label{link name}
            \sum_{i} a_i^2 = 1
        \end{equation}

        В случае, если мы не владеем полным представлением о состоянии системы, мы говорим, что она находится в смешанном состоянии. Как было сказанно выше,
        для описания смешанных систем используется оператор $\rho$, принадлежащий Гильбертову пространству, называемый матрицей плотности (или оператором 
        плотности) и задается как

        \begin{equation}\label{link name}
            \rho = \sum_i p_i \ket{\psi_i} \bra{\psi_i}
        \end{equation}

        где $p_i$ является вероятностью нахождения состояния $\psi_i$, а $\ket{\psi_i}\bra{\psi_i}$ --- соответствующий оператор проекции.
        След матрицы плотности равен 1 по условию нормировки (tr[$\rho$]=1), a сама матрица должна быть положительна, 
        по определению вероятности ($\rho > 1$). \\
        В силу утверждения (4) случае если $tr[\rho^2] = tr[\rho] = 1$ мы считаем состояние чистым. В случае $tr[\rho^2] < 1$ состояние смешанное.
        Матрица плотности представляет собой квадратную матрицу размерности $N \times N$, где N --- количество базисных векторов соответствующего 
        Гильбертова пространства.
<<<<<<< HEAD
        \\ \\
        \begin{itshape}
            Постулат 2. Пусть до измерения система находилась в чистом состоянии $\psi$.
        \end{itshape}
        \\ \\
        Согласно спектральной теореме о Эрмитовой матрице 

=======
        \\
>>>>>>> refs/remotes/origin/master
        
<<<<<<< HEAD
            






=======
>>>>>>> eb1d89c966b8765ee690b4f32be70c384eed6533
\end{document}

